\documentclass{article}
\usepackage{amsmath}
\usepackage{amssymb}
\title{MATH 4338 Main Problem 7}
\date{}
\author{Andy Lu}

\usepackage[utf8]{inputenc}
\usepackage[english]{babel}
\usepackage{amsthm}
\newtheorem{theorem*}{Question}

\begin{document}
  \maketitle
  \begin{proof}
    Suppose $a \in \mathbb{R}$ and $f'(a)$ exists at $a$. Then by the definition
    of derivative, $$\lim_{h \rightarrow 0} \frac{f(a+h) - f(a)}{h}$$ exists. Let 
    it equal $L$. By Theorem 2.8,
    $$\lim_{h \rightarrow 0} \frac{f(a+h) - f(a)}{h}  = 
      \frac{\lim_{h \rightarrow 0} \, g(a)}{\lim_{h \rightarrow 0} \, h(a)}$$,
    where $g(a) = f(a+h) - f(a)]$ and $h(a) = h$. Thus, the right hand side 
    exists. Further examining $\lim_{h \rightarrow 0} \, g(a)$, by Theorem 2.5,
    $$\lim_{h \rightarrow 0} \, g(a) = L_1 + L_2$$
    where $L_1 = \lim_{h \rightarrow 0} \, f(a+h)$ and 
    $L_2 = \lim_{h \rightarrow 0} \, f(a)$. 
    As $\lim_{h \rightarrow 0} \, f(a)$ exists, then by Theorem 2.13, the one
    sided limits exist. Since the derivative is defined at $a$, $a$ must be in
    the domain of $f$. Then by the Corollary to Theorem 2.13, $f$ is continuous
    at $a$.
    %By Theorem 2.2, $L2 = f(a)$.
    %As $f(a)$ exists from the construction of the derivative, and the limit of
    %$f(a)$ equals $f(a)$, then by the limit definition of continuity, $f$ is 
    %continuous at $a$.
  \end{proof}
\end{document}
