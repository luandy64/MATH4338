\documentclass{article}
\usepackage{amsmath}
\usepackage{amssymb}
\usepackage{mathtools}
\title{MATH 4338 Main Problem 9}
\date{}
\author{Andy Lu}

\usepackage[utf8]{inputenc}
\usepackage[english]{babel}
\usepackage{amsthm}
\newtheorem*{theorem}{Question}

\begin{document}
  \maketitle
  \begin{theorem}
    Given the function f defined on $I = {x: 0 \leq x \leq 1}$ by the formula
    \[
      f(x) =
      \begin{cases*}
        1 \quad& if $x$ is rational \\
        0 \quad& if $x$ is irrational
      \end{cases*}\]
      Prove that $\underline{\int_{0}^{1}}f(x) \, dx = 0$ and 
      $\overline{\int_{0}^{1}}f(x) \, dx = 1$.
  \end{theorem}
  \begin{proof}
    Let f be defined on $I = {x: 0 \leq x \leq 1}$ by the formula
    \[
      f(x) =
      \begin{cases*}
        1 \quad& if $x$ is rational \\
        0 \quad& if $x$ is irrational
      \end{cases*}\]
    Let $\Delta$ be a subdivision of $I$ s.t. 
    $0 = d_0 < d_1 < \ldots < d_{n-1} < d_n = 1$ with the corresponding 
    subintervals denoted $I_1, I_2, \ldots I_n$.
    By the Archimedian Principle, 
    $\forall \, I_k \, \exists$ a rational number and an irrational number in
    $I_k$, for $k = 1,2,\ldots n$. Let $m_k$ and $M_k$ denote the g.l.b and
     l.u.b of f on $I_k$. Since the range of $f$ on $I_k$ contains only 2
    values,
    \begin{equation}
      \inf_{I_k}f = 0 \qquad \sup_{I_k} f = 1
    \end{equation}
    With respect to $\Delta$,
    \begin{align*}
      S^{+}(f,\Delta) =& \sum_{i=1}^{n} M_{i}(d_i - d_{i-1})
      \tag{by definition of Upper Darboux Sum}\\
      =& \sum_{i=1}^{n} (d_i - d_{i-1}) \tag{since $M_i = 1$}\\
      =& (d_1 - d_0) + (d_2 - d_1) + \ldots + (d_n - d_{n-1})
      \tag{by definition of sum}\\
      =& d_n - d_0 \tag{by combining like terms} \\
      =& 1- 0 \tag{by construction of $\Delta$} \\
      =& 1 
    \end{align*}
    As $S^{+}(f,\Delta) = 1$, for $f$ on $I=[0,1]$ and subdivision $\Delta$, the
    g.l.b. of $S^{+}(f,\Delta)$ is $1$. Then by definition of upper Darboux
    integral, $\overline{\int_{0}^{1}}f(x) \, dx = 1$. Similarly, with respect 
    to $\Delta$,
  \end{proof}
\end{document}
