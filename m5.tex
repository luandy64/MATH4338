\documentclass{article}
\usepackage{amsmath}
\usepackage{amssymb}
\title{MATH 4338 Main Problem 5}
\date{}
\author{Andy Lu}

\usepackage[utf8]{inputenc}
\usepackage[english]{babel}
\usepackage{amsthm}
\newtheorem{theorem*}{Question}

\begin{document}
  \maketitle
  \begin{proof}
    Suppose $f$ is nondecreasing on an interval $I = \{x: a< x< b\}$, where
    $a,b \in \mathbb{R}$. Suppose $\exists M \in \mathbb{R}$ such that 
    $f(x) \leq M$ $\forall x \in I$. By definition, $M$ is an upperbound of 
    $f(x)$. By Theorem 3.5, we can pick $C \in \mathbb{R}$ such that $C$ is, by
    definition, the least upper bound of $f(x)$. Note, this means $C \leq M$. 
    Then, by the Nested Intervals Theorem, it follows that 
    $$\lim_{x \rightarrow b} f(x) = C$$. Since the limit exists, the one sided
    limit,
    $$\lim_{x \rightarrow b^-} f(x) = C$$
    must also exist.
  \end{proof}
\end{document}