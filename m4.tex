\documentclass{article}
\usepackage{amsmath}
\usepackage{amssymb}
\title{MATH 4338 Main Problem 4}
\date{}
\author{Andy Lu}

\usepackage[utf8]{inputenc}
\usepackage[english]{babel}
\usepackage{amsthm}
\newtheorem{theorem*}{Question}

\begin{document}
  \maketitle
  \begin{theorem*} 2.3 \#2
    \newline
    Given, 
    $$f(x)
    \begin{cases} 
          \frac{x^2-1}{x^4-1} & 1 < x < 2 \\
          x^2 +3x - 2 & 2 \leq x < 5
       \end{cases}
    $$
    \begin{itemize}
      \item Show whether $f$ is continuous on the left at 2 using the method from the book.
      \item Show whether $f$ is continuous on the right at 2 using the $\epsilon\text{-}\,\delta$ definition of continuity.
    \end{itemize}
  \end{theorem*}
  
  \begin{proof}
    Continuity on the left:
    
    Start by multiplying the given $f(x)$ by $1$, in the form of $\frac{x^{-4}}{x^{-4}}$:
    \begin{equation*}
      (\frac{x^2-1}{x^4-1})(\frac{x^{-4}}{x^{-4}}) = \frac{x^{-2} - x^{-4}}{1 - x^{-4}}
    \end{equation*}
    We can compute the limit of this function with the limit rules:
    \begin{align*}
      \lim_{x \rightarrow 2^-}\frac{x^{-2} - x^{-4}}{1 - x^{-4}} & &\\
      &= \frac{\lim_{x \rightarrow 2^-}(x^{-2} - x^{-4})}{\lim_{x \rightarrow 2^-}(1 - x^{-4})} & [\text{Thm 2.8}] \\
      &= \frac{\lim_{x \rightarrow 2^-}(x^{-2}) - \lim_{x \rightarrow 2^-}(x^{-4})}
              {\lim_{x \rightarrow 2^-}(1) - \lim_{x \rightarrow 2^-}(x^{-4})} & [\text{Thm 2.5}] \\
      &= \frac{\frac{1}{2^2} - \frac{1}{2^4}}{(1) - (2^{-4})} & [\lim_{x \rightarrow 2^-}(1) = 1\text{ by Thm 2.2}] \\
      &= \frac{\frac{1}{4} - \frac{1}{16}}{\frac{16}{16} - \frac{1}{16}} = \frac{3/16}{15/16} = \frac{3}{15} = \frac{1}{5} & [\text{by algebra}] \\
      & &
    \end{align*}
    As $2$ is in the domain of $f$, we can evaluate our function at $x=2$:
    \begin{equation*}
      f(2) = \frac{(2)^2-1}{(2)^4-1} = \frac{3}{15} = \frac{1}{5}
    \end{equation*}
    By the definition of continuous on the left, $f(x)$ ijs continuous on the left at $2$.
  \end{proof}
  \begin{proof}
    Continuity on the right:
    
    Let $\epsilon > 0$. Pick $\,\delta = \frac{-7 + \sqrt{49 +4\epsilon}}{2}$. \\
    Note:
    \begin{align*}
      \delta(\delta + 7) &= \epsilon \\
      \delta^2 + 7\delta - \epsilon &= 0\\
      \delta &= \frac{-7 \pm \sqrt{49 +4\epsilon}}{2} \\ 
    \end{align*}
    Since $\delta > 0$, we ignore $\frac{-7 - \sqrt{49 +4\epsilon}}{2}$. As $\epsilon > 0$, then
    $49 +4\epsilon > 49 > 0$. Taking the square root of all terms, $\sqrt{49 +4\epsilon} > 7$.
    Thus, $\,\delta > 0$ and $\delta \in \mathbb{R}$. Let $x \in [2,5)$. Suppose
    $0 \leq x-2 < \,\delta$. Then,
    \begin{align*}
      |f(x) - f(2)| &\\
                    &=|(x^2 +3x - 2) - (2^2 + 3(2) - 2)|\\
                    &=|x^2 + 3x - 2 -8|\\
                    &=|(x+5)(x-2)| \\
                    &=|x+5||x-2| \\
                    &=|x-2 + 7||x-2| <\,\delta(\,\delta + 7) = \epsilon
    \end{align*}

    Thus, by defintion, $\lim_{x \rightarrow 2+}f(x) = 2$, so $f(x)$ is continuous on the right at $2$.
  \end{proof}
\end{document}